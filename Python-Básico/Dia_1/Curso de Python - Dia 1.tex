% Created 2016-08-22 Mon 16:14
\documentclass[presentation]{beamer}
\usepackage[utf8]{inputenc}
\usepackage[T1]{fontenc}
\usepackage{fixltx2e}
\usepackage{graphicx}
\usepackage{grffile}
\usepackage{longtable}
\usepackage{wrapfig}
\usepackage{rotating}
\usepackage[normalem]{ulem}
\usepackage{amsmath}
\usepackage{textcomp}
\usepackage{amssymb}
\usepackage{capt-of}
\usepackage{hyperref}
\author{Mario Raul Freitas}
\date{\today}
\title{Dia1}
\hypersetup{
 pdfauthor={Mario Raul Freitas},
 pdftitle={Dia1},
 pdfkeywords={},
 pdfsubject={},
 pdfcreator={Emacs 25.0.94.2 (Org mode 8.3.5)}, 
 pdflang={English}}
\begin{document}

\maketitle

\section{Quem Somos Nós}
\label{sec:orgheadline2}
\subsection{CivCom}
\label{sec:orgheadline1}
\includegraphics[width=.9\linewidth]{img/Quem Somos Nós/logo2_2016-08-06_17-14-32.png}

Pesquisa, Cursos e Inovação
\section{O Que Será Dado no Curso}
\label{sec:orgheadline3}
\begin{center}
\begin{tabular}{ll}
Como Instalar Python & Como Instalar Bibliotecas\\
Prints e Inputs & Strings\\
Ints e Floats & Lists e Tuples\\
Dictionaries & Funções\\
Control Flow & Loops\\
Numpy & Matplotlib\\
Object Oriented Programming & GUI e PyQt\\
\end{tabular}
\end{center}
\section{Porque Python?}
\label{sec:orgheadline12}
Python é uma linguagem extremamente simples, mas ao mesmo tempo, muito poderas. Ela permite criar todo tipo de software, desde calculadoras a websites passando por jogos e aplicativos de celular.

Python é grátis e possui uma comunidade extremamente ativa. Há uma inifinidade de bibliotecas que facilitam a sua vida.

Python é utilizado por diversas empresas, como:
\subsection{YouTube}
\label{sec:orgheadline4}
\subsection{Dropbox}
\label{sec:orgheadline5}
\subsection{Spotify}
\label{sec:orgheadline6}
\subsection{BitTorrent}
\label{sec:orgheadline7}
\subsection{Reddit}
\label{sec:orgheadline8}
\subsection{Serpro}
\label{sec:orgheadline9}
\subsection{Frets on Fire (Guitar Hero Clone)}
\label{sec:orgheadline10}
\subsection{CivCom}
\label{sec:orgheadline11}
\section{Como Instalar Python}
\label{sec:orgheadline14}
Baixe Python 3.5.2 do site: \url{https://www.python.org/}

Alternativamente, você pode baixar o pacote miniconda de: \url{http://conda.pydata.org/miniconda.html}

Ou o pacote anaconda de: \url{https://www.continuum.io/downloads}

Recomendamos que você utilize a primeira opção
\subsection{Tutorial Disponível Em Nosso Site}
\label{sec:orgheadline13}
\url{https://civcomunb.wordpress.com/2016/07/06/hello-world/#more-36}
\section{Pip}
\label{sec:orgheadline17}
Pip é uma ferramenta do Python que serve para instalar bibliotecas disponíveis no PyPI (Python Package Index), que é um repositório oficial de biblioteca do Python.

Para utiliza-la, basta abrir o prompt de comando (cmd) e digitar:
\begin{verbatim}
pip install "nome_da_biblioteca"
\end{verbatim}

\subsection{Instalar NumPy}
\label{sec:orgheadline15}
Para instalar o NumPy digite no cmd:
\begin{verbatim}
pip install numpy
\end{verbatim}

\subsection{Intalar Matplotlib}
\label{sec:orgheadline16}
Para instalar o Matplotlib digite no cmd:
\begin{verbatim}
pip install matplotlib
\end{verbatim}

\section{Instalar PyQt4}
\label{sec:orgheadline18}
O PyQt 4 não está presente no PyPI e, portanto, não pode ser baixado pelo pip. No entanto, seu instalador pode ser  encontrado no link a seguir: \url{https://www.riverbankcomputing.com/software/pyqt/download}

Baixe e instale normalmente. Note que será necessário baixar a versão do PyQt compatível com o seu Python.
\section{Recomendação de IDE}
\label{sec:orgheadline19}
Caso deseje programar com muito mais facilidade, recomendamos que baixe o PyCharm no seguinte link: \url{https://www.jetbrains.com/pycharm/download/#section=windows}

O PyCharm é uma IDE (Integrated Development Environment), ou seja, um editor para o seu código que contém ferramentas que irão tornar a programação mais fácil. Por exemplo: auto-completar, código de cores, acesso rápido à documentação, entre outros.
\section{IDLE}
\label{sec:orgheadline20}
O IDLE possui duas janelas. Um editor de texto, onde você pode escrever seus Scripts, e um interpretador, chamado IDLE Shell.

Seus códigos sempre vão rodar em um interpretador.
\section{Hello World}
\label{sec:orgheadline22}
Com tudo instalado, agora podemos rodar nosso primeiro programa. Abra o IDLE, que deve aparecer no seu menu Iniciar após a instalção do Python. Agora digite o seguinte código e o rode:

\begin{verbatim}
print('Hello, World!')
\end{verbatim}

\begin{verbatim}
Hello, World!
\end{verbatim}


\subsection{Input}
\label{sec:orgheadline21}
\begin{verbatim}
nome = input('Qual é o seu nome? ')
print('Oi {}, como você está?'.format(nome))
\end{verbatim}

\begin{verbatim}
Oi Mario, como você está?
\end{verbatim}
\section{Variáveis}
\label{sec:orgheadline24}
Variáveis são "caixas" que guardam um tipo de informação.

Python irá criar uma "caixa" no momento em que você nomear e der valor a sua variável

Junto ao conteúdo, o Python define o tipo de conteúdo e "cola uma etiqueta" na caixa para descrever o tipo
\subsection{Regras para nomes de Variáveis}
\label{sec:orgheadline23}
Uma palavra só, sem espaços;

Deve começar com uma letra;

Não pode ser uma palavra reservada do Python.
\section{Strings}
\label{sec:orgheadline33}
String é uma variável que guarda uma série de caracteres, ou texto.

\subsection{Definindo uma String}
\label{sec:orgheadline25}
\begin{verbatim}
s1 = 'Isso é uma string'     # Isso é um comentário
s2 = 'Outra string.'
print(s1)
print(s2)
\end{verbatim}

\begin{verbatim}

>>> Isso é uma string
Outra string.
\end{verbatim}

\subsection{Concatenando Strings}
\label{sec:orgheadline26}
\begin{verbatim}
print(s1+s2)
print(s1 + '. ' + s2)
\end{verbatim}

\begin{verbatim}
Isso é uma stringOutra string.
Isso é uma string. Outra string.
\end{verbatim}


\subsection{Inserindo uma string dentro de outra com \%s}
\label{sec:orgheadline27}
\begin{verbatim}
print('Texto: %s' % s1)
print()
print('Texto 1: %s\nTexto 2: %s' % (s1, s2))
\end{verbatim}

\begin{verbatim}
Texto: Isso é uma string

Texto 1: Isso é uma string
Texto 2: Outra string.
\end{verbatim}

\subsection{Inserindo uma string dentro de outra com format}
\label{sec:orgheadline28}
\begin{verbatim}
print('Texto: {}'.format(s1))
print()
print('Texto 1: {}\nTexto 2: {}'.format(s1, s2))
\end{verbatim}

\begin{verbatim}
Texto: Isso é uma string

Texto 1: Isso é uma string
Texto 2: Outra string.
\end{verbatim}


\subsection{Múltiplos argumentos em print}
\label{sec:orgheadline29}
\begin{verbatim}
print('Texto:', s1)
\end{verbatim}

\begin{verbatim}
Texto: Isso é uma string
\end{verbatim}

\subsection{Métodos de Capitalização}
\label{sec:orgheadline30}
\begin{verbatim}
print(s2.upper())
print(s2.lower())
print(s2.capitalize())
print(s2.swapcase())
\end{verbatim}

\begin{verbatim}
OUTRA STRING.
outra string.
Outra string.
oUTRA STRING.
\end{verbatim}
\subsection{Exercício}
\label{sec:orgheadline31}
Escreva um programa que pede o nome de 3 alunos (input: str) e os salva em 3 variáveis distintas (aluno1, aluno2, aluno3). Em seguida, faça com que o programa imprima os 3 nomes seguindo a formatação mostrada abaixo: 

\begin{verbatim}
Aluno
aluno1
aluno2
aluno3
\end{verbatim}

Dica: Utilizar \n dentro de uma string pula uma linha no print (como um Enter). Alternativamente, use múltiplos prints.
\subsection{Resolução}
\label{sec:orgheadline32}
\begin{verbatim}
aluno1 = input('Digite o nome do 1o aluno: ')
aluno2 = input('Digite o nome do 2o aluno: ')
aluno3 = input('Digite o nome do 3o aluno: ')

print('Aluno')
print(aluno1)
print(aluno2)
print(aluno3)
\end{verbatim}
\section{Ints e Floats}
\label{sec:orgheadline45}
Ints são números inteiros e Floats são números reais. Operações entre ints e floats sempre resultam em floats. Divisão sempre resulta em float.
\subsection{Definindo um Int}
\label{sec:orgheadline34}
\begin{verbatim}
i1 = 5
i2 = 3
print(i1, type(i1))
\end{verbatim}

\begin{verbatim}

>>> 5 <class 'int'>
\end{verbatim}
\subsection{Definindo um Float}
\label{sec:orgheadline35}
\begin{verbatim}
f1 = 5.
f2 = 3.0
print(f1, type(f1))
\end{verbatim}

\begin{verbatim}

>>> 5.0 <class 'float'>
\end{verbatim}
\subsection{Operações - Soma e Subtração}
\label{sec:orgheadline36}
\begin{verbatim}
print(i1+i2)
print(i1-i2) 
print(i1+f1)
\end{verbatim}

\begin{verbatim}
8
2
10.0
\end{verbatim}
\subsection{Operações - Multiplicação}
\label{sec:orgheadline37}
\begin{verbatim}
print(5*3, type(5*3)) 
print(5*0.2, type(5*0.2))
\end{verbatim}

\begin{verbatim}
15 <class 'int'>
1.0 <class 'float'>
\end{verbatim}

\subsection{Operações - Divisão Real}
\label{sec:orgheadline38}
\begin{verbatim}
print(5/3, type(5/3))
print(6/3, type(6/3))
\end{verbatim}

\begin{verbatim}
1.6666666666666667 <class 'float'>
2.0 <class 'float'>
\end{verbatim}
\subsection{Operações - Divisão Inteira}
\label{sec:orgheadline39}
\begin{verbatim}
print(11//3, type(11//3))
print(11%3, type(11%3))
\end{verbatim}

\begin{verbatim}
3 <class 'int'>
2 <class 'int'>
\end{verbatim}

\subsection{Operações - Arredondamentos}
\label{sec:orgheadline40}
\begin{verbatim}
print(int(11/3))
print(round(11/3))

from math import ceil
print(ceil(10/3))
\end{verbatim}

\begin{verbatim}
3
4
>>> >>> 4
\end{verbatim}

\subsection{Operações - Exponenciação}
\label{sec:orgheadline41}
\begin{verbatim}
print(3**2)
\end{verbatim}

\begin{verbatim}
9
\end{verbatim}
\subsection{Operações - Radiciação}
\label{sec:orgheadline42}
\begin{verbatim}
from math import sqrt
print(sqrt(4))
\end{verbatim}

\begin{verbatim}

2.0
\end{verbatim}
\subsection{Exercício}
\label{sec:orgheadline43}
Adicione a seu programa a seguinte funcionalidade: pedir a matrícula (int) e nota (float) de 3 alunos e salvar em 6 variáveis distintas (mat1, mat2 ,mat3 e nota1, nota2, nota3). Em seguida, faça com que o programa imprima os nomes, matrículas e notas formatados da seguinte forma: 
\begin{verbatim}
Aluno      Matrícula    Nota
aluno1     mat1         nota1
aluno2     mat2         nota2
aluno3     mat3         nota3
\end{verbatim}

Dica: input() sempre returna uma string. Para obter int use int(input()) e para obter um float use float(input())
\subsection{Resolução}
\label{sec:orgheadline44}
\begin{verbatim}
mat1 = int(input('Matrícula 1: '))
mat2 = int(input('Matrícula 2: '))
mat3 = int(input('Matrícula 3: '))

nota1 = float(input('Nota 1: '))
nota2 = float(input('Nota 2: '))
nota3 = float(input('Nota 3: '))

print('\nAluno\tMatrícula\tNota')
print('{}\t{}\t\t{}'.format(aluno1, mat1, nota1))
print('{}\t{}\t\t{}'.format(aluno2, mat2, nota2))
print('{}\t{}\t\t{}'.format(aluno3, mat3, nota3))
\end{verbatim}
\section{Lists e Tuples}
\label{sec:orgheadline58}
Lists e Tuples são sequências de variáveis separadas por vírgulas armazendas em uma única variável. 

Lists são mutáveis, enquanto tuples são imutáveis.

Apesar de terem um uso mais limitado, tuples são mais rápidas do que lists e podem ser necessárias em certas aplicações.
\subsection{Definindo um tuple}
\label{sec:orgheadline46}
\begin{verbatim}
t1 = (0, 1, 2, 3)
t2 = ('a', 'b', 'c')
print(t1, type(t1))
\end{verbatim}

\begin{verbatim}

>>> (0, 1, 2, 3) <class 'tuple'>
\end{verbatim}

\subsection{Definindo uma list}
\label{sec:orgheadline47}
\begin{verbatim}
l1 = [0, 1, 2, 3]
l2 = ['a', 'b', 'c']
print(l2, type(l2))
\end{verbatim}

\begin{verbatim}

>>> ['a', 'b', 'c'] <class 'list'>
\end{verbatim}
\subsection{Transformando um no outro}
\label{sec:orgheadline48}
\begin{verbatim}
print(tuple(l2), type(tuple(l2)))
print(list(t1), type(list(t1)))
\end{verbatim}

\begin{verbatim}
('a', 'b', 'c') <class 'tuple'>
[0, 1, 2, 3] <class 'list'>
\end{verbatim}
\subsection{Slicing}
\label{sec:orgheadline49}
\begin{verbatim}
# l2 = ['a', 'b', 'c']
print(l2[1])
print(l2[-1])
\end{verbatim}

\begin{verbatim}

b
c
\end{verbatim}

\begin{verbatim}
#l1 = [0, 1, 2, 3]
print(t1[0:2])
print(l1[0:-1])
print('String aceita slice'[0:10])
\end{verbatim}

\begin{verbatim}

(0, 1)
[0, 1, 2]
String ace
\end{verbatim}

\begin{verbatim}
print('String aceita slice'[0:10:2])
print('String aceita slice'[::-1])
\end{verbatim}

\begin{verbatim}
Srn c
ecils atieca gnirtS
\end{verbatim}


\subsection{Fazendo Alterações com Slicing}
\label{sec:orgheadline50}
\begin{verbatim}
l2[1] = 'h'
print(l2)
\end{verbatim}

\begin{verbatim}

['a', 'h', 'c']
\end{verbatim}
\subsection{Métodos de Listas}
\label{sec:orgheadline51}
\begin{verbatim}
print(t2.index('c'))

l1.append(4)
print(l1)
\end{verbatim}

\begin{verbatim}
2
>>> >>> [0, 1, 2, 3, 4]
\end{verbatim}

\subsection{Comprimento}
\label{sec:orgheadline52}
\begin{verbatim}
print(len(t1))
print(len(l2)) 
print(len('Strings têm comprimento'))
\end{verbatim}

\begin{verbatim}
4
3
24
\end{verbatim}
\subsection{Enumerate e Zip}
\label{sec:orgheadline53}
\begin{verbatim}
alunos = ['João', 'Maria', 'Carlos']
notas = [10, 6, 8]

print(enumerate(alunos))
print(list(enumerate(alunos)))

print(zip(alunos, notas))
print(list(zip(alunos, notas)))
\end{verbatim}

\begin{verbatim}
<enumerate object at 0x00000000011D8F78>
[(0, 'João'), (1, 'Maria'), (2, 'Carlos')]
<zip object at 0x00000000011DB508>
[('João', 10), ('Maria', 6), ('Carlos', 8)]
\end{verbatim}

\subsection{Range}
\label{sec:orgheadline54}
Range é um outro tipo de variável que se parece com Tuple. Você pode gerar um range utilizando a função range(). Ela pode ser usada como range(n) ou range(n\(_{\text{i}}\), n\(_{\text{f}}\), passo). Podemos transformar o range diretamente em tuple para trabalhar com algo mais familiar.

\begin{verbatim}
print(range(3))
print(tuple(range(3)))  
print(tuple(range(2, 12, 2)))
\end{verbatim}

\begin{verbatim}
range(0, 3)
(0, 1, 2)
(2, 4, 6, 8, 10)
\end{verbatim}
\subsection{Lista de Listas (ou tuples)}
\label{sec:orgheadline55}
\begin{verbatim}
a = [[1, 2, 3], [4, 5, 6], [7, 8 ,9]]

print(a[1])
print(a[1][2])
print(sum(a[1]))
\end{verbatim}

\begin{verbatim}
[4, 5, 6]
6
15
\end{verbatim}

\subsection{Exercício}
\label{sec:orgheadline56}
Utilize os dados a seguir e crie uma lista de nomes, uma lista de  matrículas e uma lista de notas. A lista de notas será uma lista de listas em que a lista interna possui 3 elementos. Calcule a média de notas de cada aluno e imprima segundo a formatação abaixo:
\begin{verbatim}
Alunos: Alice, Beatriz, Carlos
Matrículas: 1, 2, 3
Notas:
  - Alice: 10, 7.5, 8.3
  - Beatriz: 8.8, 5.6, 5.0
  - Carlos: 3.4, 6.6, 7.7
\end{verbatim}

\begin{verbatim}
Aluno    Matrícula    Média
Alice    1            M1
Beatriz  2            M2
Carlos   3            M3
\end{verbatim}

Dica: utilize as funções sum()e len() e o método append(). Guarde as médias em uma nova lista.
\subsection{Resolução}
\label{sec:orgheadline57}
\begin{verbatim}
alunos = ['Alice', 'Beatriz', 'Carlos']
mats = [1, 2, 3]
notas = [[10, 7.5, 8.3], [8.8, 5.6, 5.0], [3.4, 6.6, 7.7]]

medias = []
medias.append(sum(notas[0])/len(notas[0]))
medias.append(sum(notas[1])/len(notas[1]))
medias.append(sum(notas[2])/len(notas[2]))

print('Alunos\t\tMatrícula\tMédia')
print('{}\t\t{}\t\t{:.1f}'.format(alunos[0], mats[0], medias[0]))
print('{}\t\t{}\t\t{:.1f}'.format(alunos[1], mats[1], medias[1]))
print('{}\t\t{}\t\t{:.1f}'.format(alunos[2], mats[2], medias[2]))
\end{verbatim}

\begin{verbatim}
Alunos          Matrícula       Média
Alice           1               8.6
Beatriz         2               6.5
Carlos          3               5.9
\end{verbatim}

\section{Dictionaries}
\label{sec:orgheadline65}
Dicionários são uma espécie de lista desorganizada que possuem a seguinte estrutura \{chave: valor\}.

Dicionários são objetos mutáveis e desorganizados.

A ordem em que os itens aparecem ao dar print é "aleatória".
\subsection{Definindo um dicionário}
\label{sec:orgheadline59}
\begin{verbatim}
d1 = {'azul': 'blue', 'rosa': 'pink', 'preto': 'black'}
d2 = {'banana': 1, 'uva': 3, 'morango': 6}
d3 = {'banana': 4, 'uva': 1, 'pera': 12}
print(d1, type(d1))
\end{verbatim}

\begin{verbatim}

>>> >>> {'rosa': 'pink', 'azul': 'blue', 'preto': 'black'} <class 'dict'>
\end{verbatim}
\subsection{Acessando um Dicionário}
\label{sec:orgheadline60}
\begin{verbatim}
print(d1['azul']) 
print(d2['banana'])
\end{verbatim}

\begin{verbatim}
blue
1
\end{verbatim}

\begin{verbatim}
item = input('Qual produto você deseja consultar o estoque? ')
print(d2[item], item + '(s)', 'no estoque')
\end{verbatim}


\subsection{Chaves e Valores}
\label{sec:orgheadline61}
\begin{verbatim}
print(d2.keys()) 
print(d2.values()) 
print(d2.items())
\end{verbatim}

\begin{verbatim}
dict_keys(['uva', 'morango', 'banana'])
dict_values([3, 6, 1])
dict_items([('uva', 3), ('morango', 6), ('banana', 1)])
\end{verbatim}

\subsection{Alterando valores}
\label{sec:orgheadline62}
\begin{verbatim}
d2['banana'] += 1
# d2['banana'] = d2['banana'] + 1 
print(d2)

d2['suco'] = 1
print(d2)
\end{verbatim}

\begin{verbatim}

... {'uva': 1, 'morango': 6, 'banana': 5, 'pera': 12}
>>> >>> {'uva': 1, 'morango': 6, 'banana': 5, 'suco': 1, 'pera': 12}
\end{verbatim}
\subsection{Exercício}
\label{sec:orgheadline63}
Utilizando os resultados do exercício anterior, armazene os nomes e médias dos alunos como \{chave:valor\} de um dicionário. Em seguida implemente uma funcionalidade em que o programa pede o nome de um aluno e retorna a média dele.

\begin{verbatim}
Aluno        Média
Alice        8.6
Beatriz      6.5
Carlos       5.9
\end{verbatim}

\begin{verbatim}
{aluno : média}
print: A média de 'fulano' foi 'tanto'
\end{verbatim}

Dica: Utilize input().
\subsection{Resolução}
\label{sec:orgheadline64}
\begin{verbatim}
alunos = ['Alice', 'Beatriz', 'Carlos']
medias = [8.6, 6.5, 5.9]

d = {alunos[0]:medias[0], alunos[1]:medias[1], alunos[2]:medias[2]}
nome = input('Digite o nome do aluno que deseja saber a média: ')

print('A média de {} foi {}'.format(nome, d[nome]))
\end{verbatim}


\section{Funções e Condicionais}
\label{sec:orgheadline78}
\subsection{Funções}
\label{sec:orgheadline66}
Funções são rotinas que tomam parâmetros de entrada, realizam um conjunto de operações e retornam um resultado. É importante resssaltar que todo e qualquer variável criada dentro da função não pode ser acessada fora dela.

Argumentos são um sinônimo para parâmetros
\subsection{Condicionais (Control Flow)}
\label{sec:orgheadline67}
Condicionais são bifurcações no código, em que se a condiação for verdadeira, o programa irá realizar um conjunto de operações e se ela for falsa o programa avança para a próxima etapa. Pode-se definir outra condicional que é testada caso a primeira seja falsa. Ainda é possível adicionar uma rotina para o caso em que todas as condicionais anteriores sejam falsas.
\subsection{Definindo uma Função}
\label{sec:orgheadline68}
\begin{verbatim}
def say_hi():
    print('Hi')

say_hi()
\end{verbatim}

\begin{verbatim}
Hi
\end{verbatim}


\subsection{Argumentos obrigatórios e opcionais}
\label{sec:orgheadline69}
\begin{verbatim}
def sum3(a, b=0, c=0):
    resp = a + b + c
    return resp

a = 10
b = 5
c = 12

print(sum3(a, b, c))
print(sum3(a, b))
print(sum3(a))
print(sum3(5, c=7))
\end{verbatim}

\begin{verbatim}
27
15
10
12
\end{verbatim}

\subsection{Lambda}
\label{sec:orgheadline70}
\begin{verbatim}
sum3_lambda = lambda a, b=0, c=0: a+b+c
print(sum3_lambda(1, 3, 5))
\end{verbatim}

\begin{verbatim}
9
\end{verbatim}
\subsection{Operadores Lógicos e Booleanos}
\label{sec:orgheadline71}
\begin{center}
\begin{tabular}{ll}
Maior que & >\\
Menor que & <\\
Igual a & ==\\
Maior ou igual a & >=\\
Menor ou igual a & <=\\
Diferente de & !=\\
Verdadeiro & True\\
Falso & False\\
E & and\\
Ou & or\\
Não (Negação) & not\\
\end{tabular}
\end{center}


\subsection{Operadores Lógicos e Booleanos - Exemplos}
\label{sec:orgheadline72}
\begin{verbatim}
print(3>2)
print(10!=4)
print(3>=10)
\end{verbatim}

\begin{verbatim}
True
True
False
\end{verbatim}

\begin{verbatim}
print(True or False)
print(True and False)
print(not True)
\end{verbatim}

\begin{verbatim}
True
False
False
\end{verbatim}
\subsection{Definido uma condicional}
\label{sec:orgheadline73}
\begin{verbatim}
def ver_dig_if(n):
    if n < 10:
        print('Número de 1 dígito')
    elif n < 100:
        print('Número de 2 dígitos')
    else:
        print('Número de 3 dígitos ou mais')

ver_dig_if(10)
\end{verbatim}

\begin{verbatim}

... ... ... ... ... ... >>> Número de 2 dígitos
\end{verbatim}

\begin{verbatim}
def ver_dig_str(n):
    n = str(n)
    n_len = len(n)
    if len == 1:
        print('Número de 1 dígito')
    else: 
        print('Número de %d dígitos' % n_len)

ver_dig_str(1245012369126401)
\end{verbatim}

\begin{verbatim}

... ... ... ... ... ... >>> Número de 16 dígitos
\end{verbatim}
\subsection{Exercício 1}
\label{sec:orgheadline74}
Faça uma função que toma uma lista de notas e calcula a média.

\begin{verbatim}
media([10, 9, 8]) >>> Resultado:  9
\end{verbatim}
\subsection{Resolução 1}
\label{sec:orgheadline75}
\begin{verbatim}
def media(l):
    m = sum(l)/len(l)
    return m

print(media([10, 9, 8]))
\end{verbatim}

\begin{verbatim}
9.0
\end{verbatim}
\subsection{Excercício 2}
\label{sec:orgheadline76}
Faça uma função que toma uma média e calcula a menção (padrão UnB).

\begin{verbatim}
média >= 9: SS
9 > média >= 7: MS
7 > média >= 5: MM
5 > média >= 3: MI
3 > média >= 0.1: II
média == 0: SR
\end{verbatim}
\subsection{Resolução 2}
\label{sec:orgheadline77}
\begin{verbatim}
def menc(media):
    if media >= 9:
        return 'SS'
    elif media >= 7:
        return 'MS'
    elif media >= 5:
        return 'MM'
    elif media >= 3:
        return 'MI'
    elif media >= 0.1:
        return 'II'
    else:
        return 'SR'

print(menc(9.5))
print(menc(7.3))
print(menc(6.1))
print(menc(4.9))
print(menc(1.3))
print(menc(0))
\end{verbatim}

\begin{verbatim}
SS
MS
MM
MI
II
SR
\end{verbatim}

\section{Loops}
\label{sec:orgheadline90}
Loops for tomam uma sequência como 'argumento' e percorrem a lista até o fim. A sequência pode ser uma lista, tuple, dicionário, range ou string

Loops while tomam um booleano (Verdadeiro ou Falso) como argumento e continaum rodando equanto o booleano for True. Booleano pode ser interpretado como uma 'condição'
\subsection{Definindo um loop for}
\label{sec:orgheadline79}
\begin{verbatim}
# Soma dos números de 1 a 10
n = 10
s = 0

for i in range(1, n+1, 1): 
    s = s + i   

print('A soma dos números de 1 a {} é {}'.format(n, s))
\end{verbatim}

\begin{verbatim}
A soma dos números de 1 a 10 é 55
\end{verbatim}

\subsection{Definindo um loop while}
\label{sec:orgheadline80}
\begin{verbatim}
# Descobre quando a soma dos números de 1 a inifinito passa de 40

senha = 'SENC16'
tentativa = ''

while tentativa != senha: 
    tentativa = input('Digite a senha: ')

print('Senha Correta!')
\end{verbatim}
\subsection{Loop em dicionário - Chaves}
\label{sec:orgheadline81}
\begin{verbatim}
d = {'node1': (0, 0), 'node2': (1, 1), 'node3': (2, 0)}
for key in d:
    print(key)

print()
for key in d.keys():
    print(key)
\end{verbatim}

\begin{verbatim}
node2
node1
node3

node2
node1
node3
\end{verbatim}
\subsection{Loop em dicionário - Valores}
\label{sec:orgheadline82}
\begin{verbatim}
d = {'node1': (0, 0), 'node2': (1, 1), 'node3': (2, 0)}
for value in d.values():
    print(value)

print()
for key in d:
    print(d[key])
\end{verbatim}

\begin{verbatim}
(2, 0)
(1, 1)
(0, 0)

(2, 0)
(1, 1)
(0, 0)
\end{verbatim}

\subsection{Loop em dicionário - Melhor Método}
\label{sec:orgheadline83}
\begin{verbatim}
d = {'node1': (0, 0), 'node2': (1, 1), 'node3': (2, 0)}

for key, value in d.items():
    print('{}:{}'.format(key, value))
\end{verbatim}

\begin{verbatim}
node2:(1, 1)
node1:(0, 0)
node3:(2, 0)
\end{verbatim}

\subsection{Loop em listas}
\label{sec:orgheadline84}
\begin{verbatim}
x = [1, 2, 3, 4, 5]
y = []

for i in x:
    y.append(i**2)

print(y)
\end{verbatim}

\begin{verbatim}
[1, 4, 9, 16, 25]
\end{verbatim}
\subsection{Loops utilizando range(len(l))}
\label{sec:orgheadline85}
\begin{verbatim}
aposta = [2, 7, 21, 33, 49, 60]
resultado = [1, 14, 21, 32, 33, 34]

corretos = 0
for i in range(len(aposta)):
    if aposta[i] in resultado:
        corretos += 1
        print('{}o  número correto ({})'.format(i+1, aposta[i]))

print('{} números corretos'.format(corretos))
\end{verbatim}

\begin{verbatim}
3o  número correto (21)
4o  número correto (33)
2 números corretos
\end{verbatim}
\subsection{Loops usando enumerate}
\label{sec:orgheadline86}
\begin{verbatim}
aposta = [2, 7, 21, 33, 49, 60]
resultado = [1, 14, 21, 32, 33, 34]

corretos = 0
for i, j  in enumerate(aposta):
    if j in resultado:
        corretos += 1
        print('{}o  número correto ({})'.format(i+1, j))

print('{} números corretos'.format(corretos))
\end{verbatim}

\begin{verbatim}
3o  número correto (21)
4o  número correto (33)
2 números corretos
\end{verbatim}

\subsection{List comprehension}
\label{sec:orgheadline87}
Em python é possível gerar listas com uma espécie de loop interno chamado list comprehension. Essa funcionalidade é bastante útil quando se quer gerar vetores que seguem uma tendência lógica.

\begin{verbatim}
list_c = [x**2 for x in range(10)]
print(list_c)
\end{verbatim}

\begin{verbatim}
[0, 1, 4, 9, 16, 25, 36, 49, 64, 81]
\end{verbatim}
\subsection{Exercício}
\label{sec:orgheadline88}
Reescrever o código do exercício de listas utilizando loops e funções. (Escrever nomes, matrículas e notas em listas, calcular as médias e imprimir formatado. ) 

\begin{verbatim}
Alunos: Alice, Beatriz, Carlos
Matrículas: 1, 2, 3
Notas:
  - Alice: 10, 7.5, 8.3
  - Beatriz: 8.8, 5.6, 5.0
  - Carlos: 3.4, 6.6, 7.7
\end{verbatim}

\begin{verbatim}
Aluno    Matrícula    Média
Alice    1            M1
Beatriz  2            M2
Carlos   3            M3
\end{verbatim}
\subsection{Resolução}
\label{sec:orgheadline89}
\begin{verbatim}
alunos = ['Alice', 'Beatriz', 'Carlos']
mats = [1, 2, 3]
notas = [[10, 7.5, 8.3], [8.8, 5.6, 5.0], [3.4, 6.6, 7.7]]

def media(l):
    m = sum(l)/len(l)
    return m

medias = []

for i in notas:
    medias.append(media(i))

print('Alunos\t\tMatrícula\tMédia')

"""
for i in range(len(alunos)): 
    print('{}\t\t{}\t\t{:.1f}'.format(alunos[i], mats[i], medias[i]))
"""

for i, j, k in zip(alunos, mats, medias):
    print('{}\t\t{}\t\t{:.1f}'.format(i, j, k))
\end{verbatim}

\begin{verbatim}
Alunos          Matrícula       Média
Alice           1               8.6
Beatriz         2               6.5
Carlos          3               5.9
\end{verbatim}

\section{Exercícios Extras}
\label{sec:orgheadline95}
\subsection{Planilha de Menções}
\label{sec:orgheadline91}
\begin{verbatim}
alunos = ['andré', 'beatriz', 'clara', 'diego', 'eduarda', 'fábio']
matriculas = [1600182, 1600945, 1600111, 1600321, 1600699, 1600099]
notas = [(2, 8, 6), (3, 10, 10), (9, 9, 8), (5, 5, 5), (10, 9, 8), (1, 2, 2)]
medias =[]
mencoes = []

# Imprimir os nomes, matrículas e notas formatados, colocando cada aluno em uma linha
# Criar função que calcula a média dos alunos e adiciona na variável médias
# Criar função que calcula a menção final dos alunos e adiciona em menção
# Imprimir Nome, Matrícula e Menção formatados, colocando cada aluno em um linha
\end{verbatim}

\subsection{Resolução}
\label{sec:orgheadline92}
\begin{verbatim}
# Parte 1
for i in range(len(alunos)):
    print('{}\t{}\t{}'.format(alunos[i], matriculas[i], notas[i]))

# Parte 2
def calc_med(notas):
    for i in notas:
        medias.append(sum(i)/3)
    print(medias)

calc_med(notas)

# Parte 3
def calc_menc(medias):
    for i in medias:
        if i >= 9:
            mencoes.append('SS')
        elif i >= 7:
            mencoes.append('MS')
        elif i >= 5:
            mencoes.append('MM')
        elif i >= 3:
            mencoes.append('MI')
        elif i > 0:
            mencoes.append('II')
        else: 
            mencoes.append('SR')
    # print(mencoes)

calc_menc(medias)

# Parte 4
for i in range(len(alunos)):
    print('{}\t{}\t{}'.format(alunos[i], matriculas[i], mencoes[i]))
\end{verbatim}
\subsection{Filtragem de Listas}
\label{sec:orgheadline93}
Use list comprehension para acessar valores da lista L\(_{\text{alpha}}\) que sejam maiores que 5 e armazená-los na lista L\(_{\text{beta}}\)

\begin{verbatim}
L_alpha = [0, 2, 3, 5, 5, 6, 12, 43, 100]
\end{verbatim}
\subsection{Resolução}
\label{sec:orgheadline94}
\begin{verbatim}
L_alpha = [0, 2, 3, 5, 5, 6, 12, 43, 100]
L_beta = [x for x in L_alpha if x > 5]
print(L_beta)
\end{verbatim}

\begin{verbatim}
[6, 12, 43, 100]
\end{verbatim}
\end{document}